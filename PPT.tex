\documentclass{article}
\usepackage{graphicx} % Required for inserting images

\title{Piedra, Papel o Tijeras}
\author{Bastian Figueroa }
\date{Mayo 2025}
\begin{document}

\maketitle

\section{Que hace?}
Este es un codigo basico que juega el clasico juego Piedra, Papel o Tijeras.


\subsection{Reglas del Juego}
El juego tiene reglas muy simples, el jugador debe elegir entre Piedra, Papel o Tijeras, la computadora eligira uno aleatoriamente tambien, el ganador sera el que cumpla los siguientes requisitos 
\newline
\begin{center}

-Sacar Piedra y tu rival saca Tijeras
\newline
  -Sacar Papel y tu rival saca Piedra
\newline
-Sacar Tijera y tu rival saca Papel
\newline
\end{center}
El algoritmo creado inicia esperando que el jugador ponga uno de los inputs ("Piedra", "Papel" o "Tijera"), luego la computadora elige uno de esos aleatoriamente, compara los inputs a traves de una serie de IF's que analizan cada regla del juego y determina cual es el ganador
\subsection{Quien lo creo?}
El creador del algoritmo fui yo, lo he programado en una tarde mientras tomaba cocacola y escuchaba system of a down a todo volumen, me lo pase muy bien, espectacular panorama, muchas gracias por la oportunidad;)
\end{document}

